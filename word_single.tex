\section{Samenvatting}
Dit document bevat de richtlijnen voor het schrijven van practicum- en projectverslagen in de Bachelor-opleiding Fysica aan de Universiteit Hasselt. Er wordt aandacht besteed aan de inhoud, de indeling en de typografische vereisten van een verslag. Onderdelen als grafieken en foutenrekening zijn van groot belang en worden dan ook extra belicht.

\section{Inleiding}
Naast wetenschappelijk leren denken, experimenteervaardigheid verwerven, modellen leren opstellen en toetsen, … is ook het schriftelijk rapporteren hierover een belangrijke vaardigheid die je in je Bachelor-opleiding moet verwerven. Het is namelijk zo dat wetenschappers hun bevindingen met anderen moeten kunnen delen, zowel mondeling – op congressen – als schriftelijk. Dit laatste doen ze door artikels te schrijven in wetenschappelijke tijdschriften. Goede verslagen leren schrijven is dus een belangrijke stap in de richting van leren communiceren met collega-fysici en andere wetenschappers.

Een goed practicumverslag rapporteert op een correcte, wetenschappelijk verantwoorde manier welke experimenten werden uitgevoerd en welke resultaten werden verkregen. Verder wordt erin uitgelegd hoe de resultaten verwerkt en geïnterpreteerd werden (welke theorie hierbij gebruikt werd, welke formules, ...). Ten slotte worden er ook besluiten uit de proeven trokken.

In principe is het uitgangspunt dat een ander persoon aan de hand van jouw verslag de proef moet over kunnen doen met een vergelijkbaar eindresultaat. Als je twijfelt of bepaalde informatie al dan niet in het verslag moet staan, komt deze regel wel van pas. Toch zijn er heel wat meer geschreven (en ongeschreven) wetten waaraan een goede wetenschappelijke publicatie moet voldoen. In deze tekst proberen we duidelijk te maken hoe je de voorbereiding VOOR, de administratie TIJDENS en de uitwerking en verslaglegging NA een practicum moet verrichten. We geven ook aanwijzingen over het maken van belangrijke onderdelen van een verslag, zoals tabellen en grafieken.
We geven ook aanwijzingen over het maken van belangrijke onderdelen van een verslag, zoals tabellen en grafieken. We geven ook aanwijzingen over het maken van belangrijke onderdelen van een verslag, zoals tabellen en grafieken. We geven ook aanwijzingen
