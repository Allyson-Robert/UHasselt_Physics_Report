\section{Theorie}
% Edit below
Men vermeldt de formules die nodig zijn om het eindresultaat af te leiden uit de metingen. 
Een korte afleiding van de formules kan ook gegeven worden als dat inzichtelijk werkt. 
Langere afleidingen welke nieuw zijn en niet uit de literatuur gehaald zijn, kun je in het appendix
opnemen. 
Als je uitgaat van onbekende of moeilijke formules is het zinvol om een literatuurverwijzing naar de afleiding ervan te geven.
Enkel een verwijzing naar een stuk theorie in het boek of informatie uit de practicumhandleiding volstaat niet: het verslag moet op zichzelf staan en alle nodige informatie bevatten.
De referentie dient alleen om aan te geven dat je de theorie niet zelf hebt afgeleid en om
geïnteresseerden de kans te geven extra informatie op te zoeken.
Hier is een voorbeeld van het citeren naar de ontwerpers van TeX \cite{latex2e} en het LaTeX handboek \cite{texbook}.
Het is zeker niet de bedoeling om de practicumhandleiding te kopiëren – beperk je tot dat wat
nodig is om de opgaven te volbrengen.
Som hier niet de formules voor fouten op, die je verder in de tekst gebruikt; deze horen thuis in
het appendix.